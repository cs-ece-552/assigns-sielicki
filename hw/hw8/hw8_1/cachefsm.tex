% Created 2019-04-12 Fri 18:55
% Intended LaTeX compiler: pdflatex
\documentclass[11pt]{article}
\usepackage[utf8]{inputenc}
\usepackage[T1]{fontenc}
\usepackage{graphicx}
\usepackage{grffile}
\usepackage{longtable}
\usepackage{wrapfig}
\usepackage{rotating}
\usepackage[normalem]{ulem}
\usepackage{amsmath}
\usepackage{textcomp}
\usepackage{amssymb}
\usepackage{capt-of}
\usepackage{hyperref}
\author{Nicholas Sielicki}
\date{\today}
\title{DM Cache Controller State Diagram}
\hypersetup{
 pdfauthor={Nicholas Sielicki},
 pdftitle={DM Cache Controller State Diagram},
 pdfkeywords={},
 pdfsubject={},
 pdfcreator={Emacs 27.0.50 (Org mode 9.2.3)}, 
 pdflang={English}}
\begin{document}

\maketitle
\tableofcontents

\begin{center}
\begin{tabular}{llrl}
\emph{Signal} & \emph{In/Out} & \emph{Width} & \emph{Description}\\
\hline
enable & In & 1 & Enable cache. Active high. If low, "write" and "comp" have no effect, and all outputs are zero.\\
index & In & 8 & The address bits used to index into the cache memory.\\
offset & In & 3 & offset[2:1] selects which word to access in the cache line. The least significant bit should be 0 for word alignment. If the least significant bit is 1, it is an error condition.\\
comp & In & 1 & Compare. When "comp"=1, the cache will compare tag\textsubscript{in} to the tag of the selected line and indicate if a hit has occurred; the data portion of the cache is read or written but writes are suppressed if there is a miss. When "comp"=0, no compare is done and the Tag and Data portions of the cache will both be read or written.\\
write & In & 1 & Write signal. If high at the rising edge of the clock, a write is performed to the data selected by "index" and "offset", and (if "comp"=0) to the tag selected by "index".\\
tag\textsubscript{in} & In & 5 & When "comp"=1, this field is compared against stored tags to see if a hit occurred; when "comp"=0 and "write"=1 this field is written into the tag portion of the array.\\
data\textsubscript{in} & In & 16 & On a write, the data that is to be written to the location specified by the "index" and "offset" inputs.\\
valid\textsubscript{in} & In & 1 & On a write when "comp"=0, the data that is to be written to valid bit at the location specified by the "index" input.\\
clk & In & 1 & Clock signal; rising edge active.\\
rst & In & 1 & Reset signal. When "rst"=1 on the rising edge of the clock, all lines are marked invalid. (The rest of the cache state is not initialized and may contain X's.)\\
createdump & In & 1 & Write contents of entire cache to memory file. Active on rising edge.\\
hit & Out & 1 & Goes high during a compare if the tag at the location specified by the "index" lines matches the "tag\textsubscript{in}" lines.\\
dirty & Out & 1 & When this bit is read, it indicates whether this cache line has been written to. It is valid on a read cycle, and also on a compare-write cycle when hit is false. On a write with "comp"=1, the cache sets the dirty bit to 1. On a write with "comp"=0, the dirty bit is reset to 0.\\
tag\textsubscript{out} & Out & 5 & When "write"=0, the tag selected by "index" appears on this output. (This value is needed during a writeback.)\\
data\textsubscript{out} & Out & 16 & When "write"=0, the data selected by "index" and "offset" appears on this output.\\
valid & Out & 1 & During a read, this output indicates the state of the valid bit in the selected cache line.\\
\end{tabular}
\end{center}


\begin{center}
\includesvg[width=.9\linewidth]{diagram}
\end{center}
\section{}
\label{sec:org83f3102}
\end{document}
